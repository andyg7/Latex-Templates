\documentclass{article} 
\usepackage{url, graphicx}
\usepackage[margin=1in]{geometry}
\usepackage{textcomp}
\usepackage{algpseudocode}
\usepackage{algorithm}
\usepackage{titling}
\usepackage{amsmath}
\usepackage{amssymb}




\title{Problem Set 1}
\author{Andrew Grant}
\date{9/8/2015}

\begin{document}

\maketitle

%First problem
\section{}
\begin{enumerate}
\item[a)] apple
\item[b)] apple
\item[c)] apple
\end{enumerate}

%Second problem
\section{}
\begin{enumerate}
\item[a)] apple
\item[b)] apple
\item[c)] apple
\end{enumerate}


%Third problem
\section{}
\begin{enumerate}
\item[a)] apple
\item[b)] apple
\item[c)] apple
\end{enumerate}


%Fourth problem
\section{}
\begin{enumerate}
\item[a)] apple
\item[b)] apple
\item[c)] apple
\end{enumerate}


%Fifth problem
\section{}
\begin{enumerate}
\item[a)] apple
\item[b)] apple
\item[c)] apple
\end{enumerate}

%MATH GENERAL
\section{Useful Math symbols}

$a^2 + b^2 = c^2$

\begin{equation}
I_D = I_F - I_R
\end{equation}

\begin{equation}
   E = mc^2 \label{clever}
 \end{equation}

\begin{equation}
  1 + 1 = 3 
 \end{equation}

\begin{equation}
  a^2 + b^2 = c^2
 \end{equation}

\begin{equation*}
  a^2 + b^2 = c^2
 \end{equation*}

Einstein equation can be found at \eqref{clever}.
\newline
\newline
\ldots when Einstein introduced his formula
\begin{equation}
e = m \cdot c^2 \;
\label{Einstein equation}
\end{equation}

In equation \eqref{Einstein equation} we have Einstein's equation again.

%SUM NOTATION
\section{Sum}
Now we have some sum stuff

\ldots these sums probably aren't correct:
\begin{equation}
\sum_{i=1}^{n} i = \frac{n * (n + 1)}{2}\;
\label{sum}
\end{equation}

\begin{equation}
  \lim_{n \to \infty}
  \sum_{k=1}^n \frac{1}{k^2}
  = \frac{\pi^2}{6}
 \end{equation}

\begin{equation}
  \sum_{k=1}^n \frac{1}{k^2}
  = \frac{\pi^2}{6}
 \end{equation}
\\
This is text style:
$\lim_{n \to \infty}
 \sum_{k=1}^n \frac{1}{k^2}
 = \frac{\pi^2}{6}$.
 
 \begin{equation}
\sum^n_{\substack{0<i<n \\
        j\subseteq i}}
   P(i,j) = Q(i,j)
\end{equation}

%POWERS
\section{Powers}

A $d_{e_{e_p}}$ mathematical
expression  followed by a
$h^{i^{g^h}}$ expression. As
opposed to a smashed
{$d_{e_{e_p}}$} expression
followed by a {$h^{i^{g^h}}$} expression.\newline

\noindent If you want to use smash \ldots \newline
A $d_{e_{e_p}}$ mathematical
expression  followed by a
$h^{i^{g^h}}$ expression. As
opposed to a smashed
\smash{$d_{e_{e_p}}$} expression
followed by a
\smash{$h^{i^{g^h}}$} expression.

%FOR ALL
\section{For all...}

$\forall x \in \mathbf{R}:
 \qquad x^{2} \geq 0$

\bigskip
\noindent $x^{2} \geq 0\qquad
 \text{for all }x\in\mathbf{R}$

\bigskip
\noindent $x^{2} \geq 0\qquad
 \text{for all } x
 \in \mathbb{R}$

%GREEKS
\section{Greeks}
\begin{center}
$\lambda,\xi,\pi,\theta,
 \mu,\Phi,\Omega,\Delta$
\end{center}

%EXPONENTS, SUBSCRIPTS, SUPERSCRIPTS
\section{Exponent, Superscripts, Subscripts}

$p^3_{ij} \qquad
 m_\text{Knuth}\qquad
\sum_{k=1}^3 k \\[5pt]
 a^x+y \neq a^{x+y}\qquad
 e^{x^2} \neq {e^x}^2$

%SQUARE ROOTS AND DOTS
\section{Square roots and dots}

$\sqrt{x} \Leftrightarrow x^{1/2}
 \quad \sqrt[3]{2}
 \quad \sqrt{x^{2} + \sqrt{y}}
 \quad \surd[x^2 + y^2]$

$\Psi = v_1 \cdot v_2
 \cdot \ldots \qquad
 n! = 1 \cdot 2
 \cdots (n-1) \cdot n$

%FUNCTIONS
\section{Functions}

$f(x) = x^2 \qquad f'(x)
 = 2x \qquad f''(x) = 2\\[5pt]
 \hat{XY} \quad \widehat{XY}
 \quad \bar{x_0} \quad \bar{x}_0$
 
 %MOD
 \section{Mod}
 
$a\bmod b \\
 x\equiv a \pmod{b}$

%FRACTIONS
\section{Fractions}

In display style:
\begin{equation*}
  3/8 \qquad \frac{3}{8}
  \qquad \tfrac{3}{8}
\end{equation*}

In text style:
$1\frac{1}{2}$~hours \qquad
$1\dfrac{1}{2}$~hours

\begin{equation*}
  \sqrt{\frac{x^2}{k+1}}\qquad
  x^\frac{2}{k+1}\qquad
  \frac{\partial^2f}
  {\partial x^2}
\end{equation*}

%ARRAYS AND MATRICES
\section{Arrays, matrices}


  \begin{equation*}
    \mathbf{X} = \left(
      \begin{array}{ccc}
        x_1 & x_2 & \ldots \\
        x_3 & x_4 & \ldots \\
        \vdots & \vdots & \ddots
      \end{array} \right)
  \end{equation*}

\bigskip
\begin{equation*}
  |x| = \left\{
    \begin{array}{rl}
      -x & \text{if } x < 0,\\
      0 & \text{if } x = 0,\\
      x & \text{if } x > 0.
    \end{array} \right.
\end{equation*}

\bigskip
\begin{equation*}
  \begin{matrix}
1 & 2 \\
3&4 \end{matrix} \qquad \begin{bmatrix}
    p_{11} & p_{12} & \ldots
    & p_{1n} \\
    p_{21} & p_{22} & \ldots
    & p_{2n} \\
    \vdots & \vdots & \ddots
    & \vdots \\
    p_{m1} & p_{m2} & \ldots
    & p_{mn}
  \end{bmatrix}
\end{equation*}

\section{Algo Section}

Creating nice pseudocode algorithms.

\subsection{My Algorithm}

Did you learn \LaTeX?

\begin{algorithmic}
\If {$learned \geq sortof$}
	\State $okay \gets true$
\Else
	\While {$!okay$}
		\State repeat course
		\State {$okay = learned \geq sortof$}
	\EndWhile
\EndIf
\end{algorithmic}

\section{Footnotes}
Trying to create footnotes \footnote{This is a footnote.} as I learn \LaTeX.

\section{Emphasized words}
I'm working on \emph{emphasizing} and adding \underline{underlines} to all my \LaTeX{} documents.

\section{Lists and stuff}
\begin{enumerate}
\item first list
\begin{itemize}
\item What 
\item a 
\item cool
\item list.  
\end{itemize}
\item second list
\begin{itemize}
\item Now 
\item another
\item one
\item bro 
\end{itemize}
\end{enumerate}

\begin{center}
At the centre of \\ the earth.
\end{center}


\section{Conclusion}
We can conclude by referencing a bunch of our equations. Below we have $24 * 60 = 1,440$ in equation~\ref{eq:minutes}.
\begin{equation}
24 * 60 = 1,440 
\label{eq:minutes}
\end{equation}




\end{document} 
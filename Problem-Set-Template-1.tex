\documentclass{article} 
\usepackage{url, graphicx}
\usepackage[margin=1in]{geometry}
\usepackage{textcomp}
\usepackage{algpseudocode}
\usepackage{algorithm}
\usepackage{titling}
\usepackage{amsmath}
\usepackage{amssymb}



\title{Problem Set 1}
\author{Andrew Grant}
\date{9/8/2015}

\begin{document}

\maketitle

%First problem
\section{}
\begin{enumerate}
\item[a)] apple
\item[b)] apple
\item[c)] apple
\end{enumerate}

%Second problem
\section{}
\begin{enumerate}
\item[a)] apple
\item[b)] apple
\item[c)] apple
\end{enumerate}


%Third problem
\section{}
\begin{enumerate}
\item[a)] apple
\item[b)] apple
\item[c)] apple
\end{enumerate}


%Fourth problem
\section{}
\begin{enumerate}
\item[a)] apple
\item[b)] apple
\item[c)] apple
\end{enumerate}


%Fifth problem
\section{}
\begin{enumerate}
\item[a)] apple
\item[b)] apple
\item[c)] apple
\end{enumerate}

\section{Useful Math symbols}

$a^2 + b^2 = c^2$
\begin{equation}
   E = mc^2 \label{clever}
 \end{equation}

\begin{equation}
  1 + 1 = 3 
 \end{equation}

\begin{equation}
  a^2 + b^2 = c^2
 \end{equation}

\begin{equation*}
  a^2 + b^2 = c^2
 \end{equation*}

Einstein equation can be found at \eqref{clever}.
\newline
\newline
\ldots when Einstein introduced his formula
\begin{equation}
e = m \cdot c^2 \;
\label{Einstein equation}
\end{equation}

In equation \eqref{Einstein equation} we have Einstein's equation again.

%Sum notation
\section{Sum}
Now we have some sum stuff

\ldots these sums probably aren't correct:
\begin{equation}
\sum_{i=1}^{n} i = \frac{n * (n + 1)}{2}\;
\label{sum}
\end{equation}

\begin{equation}
  \lim_{n \to \infty}
  \sum_{k=1}^n \frac{1}{k^2}
  = \frac{\pi^2}{6}
 \end{equation}

\begin{equation}
  \sum_{k=1}^n \frac{1}{k^2}
  = \frac{\pi^2}{6}
 \end{equation}
\\
This is text style:
$\lim_{n \to \infty}
 \sum_{k=1}^n \frac{1}{k^2}
 = \frac{\pi^2}{6}$.

%Powers
\section{Powers}

A $d_{e_{e_p}}$ mathematical
expression  followed by a
$h^{i^{g^h}}$ expression. As
opposed to a smashed
{$d_{e_{e_p}}$} expression
followed by a {$h^{i^{g^h}}$} expression.\newline

\noindent If you want to use smash \ldots \newline
A $d_{e_{e_p}}$ mathematical
expression  followed by a
$h^{i^{g^h}}$ expression. As
opposed to a smashed
\smash{$d_{e_{e_p}}$} expression
followed by a
\smash{$h^{i^{g^h}}$} expression.

%For all
\section{For all...}

$\forall x \in \mathbf{R}:
 \qquad x^{2} \geq 0$

\bigskip
\noindent $x^{2} \geq 0\qquad
 \text{for all }x\in\mathbf{R}$

\bigskip
\noindent $x^{2} \geq 0\qquad
 \text{for all } x
 \in \mathbb{R}$

%Greeks
\section{Greeks}

$\lambda,\xi,\pi,\theta,
 \mu,\Phi,\Omega,\Delta$



\begin{equation}
I_D = I_F - I_R
\end{equation}

\section{Algo Section}

Creating nice pseudocode algorithms.

\subsection{My Algorithm}

Did you learn \LaTeX?

\begin{algorithmic}
\If {$learned \geq sortof$}
	\State $okay \gets true$
\Else
	\While {$!okay$}
		\State repeat course
		\State {$okay = learned \geq sortof$}
	\EndWhile
\EndIf
\end{algorithmic}

\section{Footnotes}
Trying to create footnotes \footnote{This is a footnote.} as I learn \LaTeX.

\section{Emphasized words}
I'm working on \emph{emphasizing} and adding \underline{underlines} to all my \LaTeX{} documents.

\section{Lists and stuff}
\begin{enumerate}
\item first list
\begin{itemize}
\item What 
\item a 
\item cool
\item list.  
\end{itemize}
\item second list
\begin{itemize}
\item Now 
\item another
\item one
\item bro 
\end{itemize}
\end{enumerate}

\begin{center}
At the centre of \\ the earth.
\end{center}


\section{Conclusion}
We can conclude by referencing a bunch of our equations. Below we have $24 * 60 = 1,440$ in equation~\ref{eq:minutes}.
\begin{equation}
24 * 60 = 1,440 
\label{eq:minutes}
\end{equation}




\end{document} 
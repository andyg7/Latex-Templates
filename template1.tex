\documentclass{article} 
\usepackage{url, graphicx}
\usepackage[margin=1in]{geometry}
\usepackage{textcomp}
\usepackage{algpseudocode}
\usepackage{algorithm}


\title{Problem Set 1}
\author{Andrew Grant}
\date{9/8/2015}

\begin{document}
\tableofcontents

\maketitle

\section{First Section}

Today is \today. \newline
``Hey man what is $5 - 1$?''\newline
This is a small new parapgraph that doesn't contain much useful info. Just another \LaTeX{} doc.
Books are often typeset with each line having the same length. LATEX 
inserts the necessary line breaks and spaces between words by optimizing the contents of a whole paragraph. If necessary, it also hyphenates words that would not fit comfortably on a line. How the paragraphs are typeset depends on the document class. Normally the first line of a paragraph is indented, and there is no additional space between two paragraphs. Refer to section 6.3.2 for more information.
Not a new line? 
% This is an epic comment

Is it cool to have a new paragraph here or not? Read\slash{} write

\subsection{Temp info}

Temp today  \today{} is $-30\, ^{\circ}\mathrm{C}$

Another way to write it is as 30 \textcelsius{} is 86 \textdegree{}F

\section{Seconds Section}

In equation~\ref{Einstein equation} we have Einstein's equation and in equation~\ref{sum} we have the sum notation.

%Example 1
\ldots when Einstein introduced his formula
\begin{equation}
e = m \cdot c^2 \;
\label{Einstein equation}
\end{equation}

%Example 2
\ldots from which follows Kirchhoff's current law:
\begin{equation}
\sum_{k=1}^{n} I_k = 0\;
\label{sum}
\end{equation}

%Example 3
\ldots which has several advantages

\begin{equation}
I_D = I_F - I_R
\end{equation}

\section{Algo Section}

Creating nice pseudocode algorithms.

\subsection{My Algorithm}

Did you learn \LaTeX?

\begin{algorithmic}
\If {$learned \geq sortof$}
	\State $okay \gets true$
\Else
	\While {$!okay$}
		\State repeat course
		\State {$okay = learned \geq sortof$}
	\EndWhile
\EndIf
\end{algorithmic}

\section{Footnotes}
Trying to create footnotes \footnote{This is a footnote.} as I learn \LaTeX.

\section{Emphasized words}
I'm working on \emph{emphasizing} and adding \underline{underlines} to all my \LaTeX{} documents.

\section{Lists and stuff}
\begin{enumerate}
\item first list
\begin{itemize}
\item What 
\item a 
\item cool
\item list.  
\end{itemize}
\item second list
\begin{itemize}
\item Now 
\item another
\item one
\item bro 
\end{itemize}
\end{enumerate}

\begin{center}
At the centre of \\ the earth.
\end{center}


\section{Conclusion}
We can conclude by referencing a bunch of our equations. Below we have $24 * 60 = 1,440$ in equation~\ref{eq:minutes}.
\begin{equation}
24 * 60 = 1,440 
\label{eq:minutes}
\end{equation}




\end{document} 